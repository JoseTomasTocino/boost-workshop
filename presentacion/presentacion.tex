\documentclass[8pt,xcolor=svgnames]{beamer}
\usepackage{paquetes}
\usepackage{modo}

\setlength{\parskip}{0.4cm}

\begin{document}

% Transparencia de Inicio -> Título
\begin{frame}
  \titlepage
\end{frame}

\normalsize

% Transparencia de índice
\begin{frame}
  \frametitle{Índice} 
  % \transboxin
  \tableofcontents
\end{frame}


\section{Introducción}

\subsection{Disclaimers iniciales}

\begin{frame}{Disclaimers iniciales}
  \begin{block}{Sobre ti, estimado miembro de la audiencia}
    \begin{itemize}
    \item Deberías saber algo de C++.
    \item Deberías saber hacer Makefiles o a compilar en la línea de comandos.
    \end{itemize}     
  \end{block}

  \pause

  \begin{block}{Sobre mi, oh todopoderoso ponente}
    \begin{itemize}
    \item No soy un experto en Boost, ni mucho menos.
    \item No soy un experto programador, ese derecho solo lo tiene Gerardo.
    \item Este es mi primer taller, \textit{so bear with me}.
    \end{itemize}
  \end{block}

  \pause
  \begin{alertblock}{Total disclaimer 1.0}
    Según la Wikipedia y el dRAE:
    \[
    \textrm{librería} \simeq \textrm{biblioteca} 
    \]
    en este contexto.
  \end{alertblock}
\end{frame}

\subsection{Qué es Boost}

\begin{frame}{¿Qué es Boost?}
  Conjunto de bibliotecas de código abierto y revisión por pares para C++.
  \begin{itemize}
  \item Pensadas para la productividad.
  \item Más de 80 bibliotecas.
  \item Escritas por los mejores programadores C++. Revisión por pares.
  \item Nueva versión cada tres meses.
  \item 12 de estas librerías forman parte del C++ Technical Report 1:
    Array, Bind, Function, Hash, mem\_fn, Random, Ref, Regex, Smart
    Pointers, Tuple, Type Traits y Unordered.
  \end{itemize}
\end{frame}

\begin{frame}{Compilando con Boost}
  \begin{block}{¿Cómo compilo Boost?}
    Bueno, en realidad... en la mayoría de los casos no hay
    \textbf{nada que compilar}. La inmensa mayoría de librerías de
    boost se alojan en ficheros de cabecera que no hace falta
    compilar. Simplmente se incluyen en tu proyecto.
  \end{block}

  \begin{block}{Algunas excepciones}
    Hay bibliotecas de boost que por su complejidad necesitan ser
    compiladas y enlazadas:
    {\scriptsize
      \begin{itemize}
      \item Boost.Filesystem
      \item Boost.IOStreams
      \item Boost.ProgramOptions
      \item Boost.Python
      \item Boost.Regex
      \item Boost.Serialization
      \item Boost.Signals
      \item Boost.System
      \item Boost.Thread
      \item Boost.Wave
      \end{itemize}
    }
  \end{block}
  
\end{frame}

\begin{frame}{Organización del taller}
  Este taller se va a organizar en dos partes:
  \begin{enumerate}
  \item En la primera parte vamos a hacer un repaso por las librerías
    que forman parte de TR1, explicando brevemente qué hace cada una y
    poniendo un ejemplo. Algunas de estas librerías se explicarán con
    más detenimiento en la segunda parte del taller.
    \pause
  \item Para la segunda parte he hecho una selección de las librerías
    que a mi juicio son más prácticas e interesantes. Para cada una de
    ellas el esquema que seguiré es el siguiente:
    \pause
    \begin{enumerate}
    \item Expondré un problema de diseño/programación y dejaré unos
      minutos para que penséis una solución con vuestros conocimientos
      \textit{non-boost-powered}\copyright.
      \pause
    \item Tras ese tiempo, pondremos en común las posibles soluciones,
      que a buen seguro serán burdas aproximaciones de lo que sería
      posible con Boost.
      \pause
    \item Presentaré y explicaré una biblioteca de Boost con la que
      podríamos abordar el problema.
      \pause
    \item Finalmente, resolveré el problema inicial usando Boost, y
      algún otro ejemplo.
    \end{enumerate}
  \end{enumerate}
  
\end{frame}

\section{Repaso a las bibliotecas en TR1}

\begin{frame}{¿Qué es TR1?}
  \begin{block}{¿Qué es TR1?}
    C++ Technical Report 1: documento que propone una serie de
    adiciones a la biblioteca estándard de C++. No es un estándard
    oficial, sino un borrador, aunque es muy probable que forme parte
    del siguiente estándard \texttt{C++0x}.
  \end{block}
  \begin{block}{}
    Al ser una ampliación de la biblioteca estándard, no modifica
    aspectos base de C++ como sí hará \texttt{C++0x}, simplemente
    añade componentes.
  \end{block}
\end{frame}

\begin{frame}[fragile]{Boost.Array}
  \begin{block}{}
    \texttt{boost/array.hpp}
  \end{block}

  \begin{block}{boost::array}
    Wrapper para arrays de tamaño fijo (típicas de C), además de
    cumplir con las especificaciones de contenedor de la STL.
    
    \medskip
    
    Muy parecida a \texttt{std::vector}, pero con tamaño fijo, y
    capacidad de ser definido como un array tradicional:
    \begin{lstlisting}[style=C++]
      boost::array<string, 3> primos = { {"Shurmano", "Shurprimo", "Shurperro"} };
    \end{lstlisting}
  \end{block}
  \pause
  \begin{block}{}
    Veamos el ejemplo en \texttt{ejemplos\_tr1/boost\_array}
  \end{block}
  \pause
  \begin{block}{Información y reservas}
    \begin{itemize}
    \item \url{http://www.boost.org/doc/libs/1_43_0/doc/html/array.html}
    \item \url{http://std.dkuug.dk/jtc1/sc22/wg21/docs/papers/2003/n1548.htm}
    \end{itemize}
    
  \end{block}
\end{frame}

% \subsubsection{Boost.Bind}

\begin{frame}[fragile]{Boost.Bind}
  \begin{block}{}
    \textit{bind: } envolver, vendar, atar, amarrar, ligar.
  \end{block}
  \begin{block}{boost::bind}
    \texttt{boost::bind} es una generalización de \texttt{std::bind1st} y
    \texttt{std::bind2nd}. Nos permite \textit{bindear} una serie de
    argumentos a cualquier cosa que \textit{tenga pinta de función}:
    punteros a función, objetos función, y punteros a función-miembro.
  \end{block}

  \pause
  \begin{center}
    {\huge What?}
  \end{center}

  \pause

  \begin{block}{}
    La veremos en la segunda parte del taller.
  \end{block}
\end{frame}

% \subsubsection{Boost.Function}

\begin{frame}{Boost.Function}
  \begin{block}{boost::function}
    \texttt{boost::function} nos proporciona \textit{alojamiento} de funciones,
    funciones objeto y funciones miembro para su posterior
    ejecución. Se suele utilizar en conjunto con Boost::bind.
  \end{block}

  \pause

  \begin{block}{}
    La veremos en la segunda parte del taller.
  \end{block}  
\end{frame}

% \subsubsection{Boost.Hash}

\begin{frame}{Boost.Hash}
  \begin{block}{}
    \texttt{boost/functional/hash.hpp}
  \end{block}
  \begin{block}{boost::hash}
    \texttt{boost::hash} nos permite generar hashes para enteros, flotantes,
    punteros y cadenas de forma nativa, y además provee soporte para
    la generación de hashes para tipos de datos definidos por el
    usuario y contenedores.
  \end{block}
  \pause
  \begin{block}{}
    Veamos el ejemplo en \texttt{ejemplos\_tr1/boost\_hash}
  \end{block}
\end{frame}

% \subsubsection{Boost.Mem\_fn}

\begin{frame}{Boost.Mem\_fn} 
  \begin{block}{}
    \texttt{boost/mem\_fn.hpp}
  \end{block}
  \begin{block}{boost::mem\_fn}
    \texttt{boost::mem\_fn} se podría definir como un caso concreto de
    \texttt{boost::bind}. Sirve para generar objetos función que
    apuntan a funciones miembro. De ahí el nombre: mem\_fn = member
    function.
  \end{block}
  \pause
  \begin{block}{Acerca del solapamiento con bind}
    \url{http://stackoverflow.com/questions/3088058/whats-the-point-of-using-boostmem-fn-if-we-have-boostbind}
  \end{block}
  \pause
  \begin{block}{}
    Veamos el ejemplo en \texttt{ejemplos\_tr1/boost\_mem\_fn}
  \end{block}  
\end{frame}

% \subsubsection{Boost.Random}

\begin{frame}{Boost.Random}
  \begin{block}{}
    \texttt{boost/random/*}
  \end{block}
  \begin{block}{boost::random}
    \texttt{boost::random} sirve, como su nombre indica, para trabajar
    con números aleatorios. Proporciona una serie de generadores
    aleatorios, de diferente eficiencia, y un conjunto de
    distribuciones a las que podemos mapear la salida de los
    generadores para facilitar las cosas, en lugar de andar nosotros
    haciendo el módulo y tal.
  \end{block}  
  \pause
  \begin{block}{}
    Veamos el ejemplo en \texttt{ejemplos\_tr1/boost\_random}
  \end{block}  
\end{frame}

% \subsubsection{Boost.Ref}

\begin{frame}{Boost.Ref}
  \begin{block}{}
    \texttt{boost/functional/ref.hpp}
  \end{block}
  \begin{block}{boost::ref}
    \texttt{boost::ref} sirve para pasar referencias a funciones que
    normalmente tomarían sus argumentos por valor. Así evitamos la
    duplicación, que en algunos casos no es posible.
  \end{block}
  \begin{alertblock}{Disclaimer}
    No he podido dedicarle tiempo, así que no tengo mucha idea del
    tema.
  \end{alertblock}
\end{frame}

% \subsubsection{Boost.Regex}

\begin{frame}{Boost.Regex}
  \begin{block}{boost::regex}
    \texttt{boost::regex} es una biblioteca de expresiones regulares
    que veremos en la segunda parte del taller.
  \end{block}
\end{frame}

% \subsubsection{Boost Smart Pointers}

\begin{frame}{Boost Smart Pointers}
  \begin{block}{Boost Smart Pointers}
    Los \texttt{smart pointers} de boost son una serie de punteros
    inteligentes que se ocupan automáticamente de la gestión de la
    vida de los objetos, ahorrándonos tener que liberar la memoria al
    terminar de usarlos
  \end{block}
  \begin{block}{}
    La veremos en la segunda parte del taller.
  \end{block}  
\end{frame}

% \subsubsection{Boost.Tuple}

\begin{frame}{Boost.Tuple}
  \begin{block}{}
    \texttt{boost/tuple/*}
  \end{block}
  \begin{block}{boost::tuple}
    \texttt{boost::tuple} es una especie de versión generalizada de
    \texttt{std::pair}, en la que se puede almacenar un número
    arbitrario de valores de distinto tipo. Es especialmente útil
    cuando necesitamos que una función devuelva varios valores.
  \end{block}
  \pause
  \begin{block}{}
    Veamos el ejemplo en \texttt{ejemplos\_tr1/boost\_tuple}
  \end{block}  
\end{frame}

% \subsubsection{Boost.TypeTraits}

\begin{frame}{Boost.TypeTraits}
  \begin{block}{}
    \texttt{boost/type\_traits.hpp}
  \end{block}
  \begin{block}{Background}
    Una \texttt{trait class} es una clase paramétrica que ofrece información
    sobre un tipo de datos en tiempo de compilación. (trait = rasgo)
  \end{block}
  \pause
  \begin{block}{boost.typetraits}
    Se trata de una colección de clases trait que nos ayudarán a la
    hora de caracterizar un tipo de datos cuando trabajemos con
    programación genérica, pudiendo adaptar nuestros algoritmos según
    las propiedades del tipo de datos pero manteniendo la genericidad.
  \end{block}
  \pause
  \begin{block}{}
    Veamos el ejemplo en \texttt{ejemplos\_tr1/boost\_typetraits}
  \end{block}  
\end{frame}

% \subsubsection{Boost.Unordered}

\begin{frame}{Boost.Unordered}
  \begin{block}{}
    \texttt{boost.unordered} proporciona alternativas no ordenadas a
    \texttt{set}, \texttt{map}, \texttt{multiset} y \texttt{multimap},
    basándose en \texttt{boost::hash} para crear una tabla hash,
    mejorando el rendimiento en el acceso a los elementos.
  \end{block}
  \pause
  \begin{block}{boost/unordered\_set.hpp}
    Proporciona los tipos \texttt{boost::unordered\_set} y
    \texttt{unordered\_multiset}.
  \end{block}
  \pause
  \begin{block}{boost/unordered\_map.hpp}
    Proporciona los tipos \texttt{boost::unordered\_map} y
    \texttt{unordered\_multimap}.
  \end{block}
  \pause
  \begin{block}{}
    Veamos el ejemplo en \texttt{ejemplos\_tr1/boost\_unordered}
  \end{block}  
\end{frame}

\section{Segunda parte: bibliotecas útiles de Boost}
\subsection{boost::format}

\begin{frame}{boost::format}
  \begin{block}{}
    \texttt{boost/format.hpp}
  \end{block}
  \begin{block}{boost::format}
    Nos permite formatear cadenas utilizando la sintaxis típica de printf.

    \begin{itemize}
    \item Utiliza un flujo interno en el que vuelca los argumentos
    \item Utiliza el operador \% para pasar los valores.
    \end{itemize}
  \end{block}

  \begin{block}{Sintaxis básica}
    \texttt{boost::format( cadena-formato ) \% arg1 \% arg2 \% ... \% argN}    
  \end{block}

  \begin{block}{También como objeto}
    \texttt{boost::format miFormato(" ... ");} \\
    \texttt{miFormato \% valor1 \% valor2; }
  \end{block}
\end{frame}

\begin{frame}{boost::format}
  \begin{block}{Sintaxis de la cadena de formato}
    Compuesta de bloques que definen el formato de cada argumento:
    \begin{itemize}
    \item \texttt{\%spec} Forma clásica de printf.
    \item \texttt{\%|spec|}  Como la anterior pero no es necesario indicar el tipo de datos.
    \item \texttt{\%N\%} Forma simple, sólo se indica la posición del argumento.
    \end{itemize}    
  \end{block}

  \begin{block}{Sintaxis de \texttt{spec}}
    \texttt{[ N\$ ][ flags ][ ancho ][ .precisión ][caracter-de-tipo]}    
    \begin{itemize}
    \item \texttt{[ N\$ ]} Referencia al argumento N-ésimo.
    \item \texttt{[ flags ]} Posibles banderas de formato:
      {\scriptsize
        \begin{itemize}
        \item \texttt{'-'}  alineación a la izquierda.
        \item \texttt{'='}  alineación centrada.
        \item \texttt{'\_'}  alineación interna.
        \item \texttt{'+'}  mostrar signo incluso para números positivos.
        \item \texttt{'0'}  rellenar con ceros.
        \end{itemize}
      }
    \item \texttt{[ ancho ]} ancho mínimo. Relleno con espacios.
    \end{itemize}
  \end{block}
\end{frame}

\begin{frame}[fragile]{boost::format}
  \begin{block}{Tabulación absoluta}
    \begin{itemize}
    \item \texttt{\%|nt|} siendo \texttt{n} un entero, indica que se
      utilizarán caracteres de relleno si es necesario para que la cadena
      esté a \texttt{n} caracteres del inicio.
    \item \texttt{\%|nTX|} igual que en el caso anterior, pero X indicará
      el caracter de relleno.
    \end{itemize} 
    Se escribe antes de cada cadena \texttt{spec}.
  \end{block}
\end{frame}

\begin{frame}[fragile]{boost::format Ejemplos}
  \begin{block}{}
    {\scriptsize
\begin{verbatim}
// argumentos por posición
// format("%|2$| %1$i \n") % "11" % "hola";

hola 11

// alineación centrada, ancho 15
// format("|%|=15||%|=15||%|=15||\n") % "Oficina" % "Software" % "Libre";

|    Oficina    |    Software   |     Libre     |

// Tabulación absoluta
// format("%2% %|10t|%|1$| \n") % "Godofredo" % "A1";    
// format("%2% %|10t|%|1$| \n") % "Godofredo" % "A12345";

A1        Godofredo 
A12345    Godofredo

// Notación de números
// format("%|30.5| \n") % 12312344.42345823904;
// format("%|+30.5| \n") % 12312344.42345823904;  
// format("%|+30.5e| \n") % 12312344.42345823904; 
// format("%|+30.5f| \n") % 123123.42345823904;   

                    1.2312e+07 
                   +1.2312e+07 
                  +1.23123e+07 
                 +123123.42346 
\end{verbatim}
      
    }
  \end{block}
\end{frame}

\subsection{boost::regex}
\begin{frame}{boost::regex}
  \begin{block}{}
    \texttt{boost/regex.hpp}
  \end{block}
  \begin{block}{Expresiones regulares}
    Las expresiones regulares son unas expresiones que sirven para
    desribir conjuntos de cadenas. Utilizan una sintaxis bastante
    normalizada, y su utilidad se centra en buscar y manipular textos.
  \end{block}
  \pause
  \begin{block}{boost::regex}
    Ofrece soporte para búsquedas y reemplazos en textos usando
    expresiones regulares de manera muy sencilla e intuitiva.
  \end{block}
  \pause
  \begin{alertblock}{Enlazado}
    Es una de las pocas bibliotecas de Boost que necesita de
    enlazado: \texttt{-lboost\_regex}
  \end{alertblock}
\end{frame}

\begin{frame}[fragile]{Tipos de datos}
  \begin{block}{boost::regex}
    \texttt{boost::regex} se usa para definir la expresión
    regular. Viene en dos versiones: \texttt{regex} y \texttt{wregex}.
\begin{verbatim}
       boost::regex patron("\\w+(\\s+\\w+){2,}\\s*");
\end{verbatim}
  \end{block}

  \pause

  \begin{block}{boost::match\_results}
    \texttt{match\_results} es una clase paramétrica que guarda
    información sobre las ocurrencias encontradas al realizar una
    operación con una expresión regular.
    
    \begin{itemize}
    \item \texttt{cmatch} para char *
    \item \texttt{wcmatch} para wchar *
    \item \texttt{smatch} para string
    \item \texttt{wsmatch} para wstring
    \end{itemize}
  \end{block}
\end{frame}

\begin{frame}[fragile]{Funciones libres}
  \begin{block}{Coincidencia total}
    \texttt{boost::regex\_match} nos sirve para ver si una cadena
    concuerda \textbf{en su totalidad} con una expresión regular.
\begin{verbatim}
bool boost::regex_match(cadena, resultado, patrón);
\end{verbatim}
  \end{block}

  \pause
  \begin{block}{Coincidencia parcial}
    \texttt{boost::regex\_search} se usar para buscar alguna
    coincidencia de la expresión regular dentro de una cadena.
\begin{verbatim}
bool boost::regex_search(cadena, resultado, patrón);
\end{verbatim}
  \end{block}
  
  \pause
  \begin{block}{Reemplazo}
    \texttt{boost::regex\_replace} reemplaza ocurrencias de la
    expresión regular en la cadena con otra en el formato que
    indiquemos.
    
\begin{verbatim}
string boost::regex_replace(cadena, patron, reemplazo);      
\end{verbatim}

  \end{block}

\end{frame}

\begin{frame}{Iteradores}
  \begin{block}{boost::regex\_iterator}
    \texttt{boost::regex\_iterator} nos permitará iterar sobre las
    ocurrencias de una expresión regular sobre una cadena.
  \end{block}
  
  \pause

  \begin{block}{boost::regex\_token\_iterator}
    \texttt{boost::regex\_token\_iterator} nos puede servir para
    tokenizar una cadena, usando un patrón como divisor.
  \end{block}

  \pause
  \begin{block}{}
    Ambos iteradores vienen en cuatro sabores, según el tipo de
    caracter que se use. En nuestros ejemplos usaremos
    \texttt{std::string}, así que habría que anteponer una \textit{s}
    a los nombres de los tipos
  \end{block}
\end{frame}

\subsection{boost::program\_options}

\begin{frame}{boost::program\_options}
  \begin{block}{}
    \texttt{boost/program\_options.hpp}
  \end{block}
  \begin{block}{boost::program\_options}
    \texttt{boost::program\_options} nos permite parsear, de manera
    sencilla, efectiva y segura, los argumentos que le pasemos a los
    programas al ejecutarlos en la línea de comandos, así como parsear
    ficheros de configuración y variables de entorno.
  \end{block}
  \pause
  \begin{alertblock}{Enlazado}
    Es otra de las pocas bibliotecas de Boost que necesita de
    enlazado: \texttt{-lboost\_program\_options}
  \end{alertblock}
\end{frame}

\begin{frame}{Modularización}
  \begin{block}{Modularización}
    El proceso de parseo tiene tres partes:
    \begin{itemize}
    \item Descripción sintáctica y semántica de las opciones y argumentos.
    \item Parseo de la entrada, ya sea de la línea de comandos o de otro origen.
    \item Almacenamiento en un mapa de variables o usando referencias.
    \end{itemize}
  \end{block}  
\end{frame}

\begin{frame}{Definición de opciones}
  \begin{block}{}
    \texttt{namespace po = boost::program\_options;}
  \end{block}
  \begin{block}{Opciones}
    Los objetos \texttt{po::options\_description} guardan la
    descripción de todas las opciones. Éstas suelen tener tres partes:
    \begin{itemize}
    \item \textbf{Sintaxis:} Nombre corto y largo.
    \item \textbf{Semántica:} ¿Recibe valores?
    \item \textbf{Descripción:} Información sobre la opción.
    \end{itemize}
  \end{block}
  \pause
  \begin{block}{Opciones posicionales}
    Hay veces en las que incluímos argumentos que no pertenecen a
    ninguna opción, sino que \textit{van por libre}. Los veremos en el
    segundo ejemplo.
  \end{block}
\end{frame}

\begin{frame}{Parseo y almacenamiento}
  \begin{block}{Almacenamiento}
    Se utilizará un tipo \texttt{po::variables\_map} para guardar las
    opciones parseadas, que es un mapa multivalor (hace uso de
    \texttt{boost::any}).
  \end{block}
  \pause
  \begin{block}{Parseo}
    Si no tenemos grandes pretensiones, podemos utilizar la función
    libre \texttt{po::parse\_command\_line}, que nos devolverá las
    opciones parseadas que podremos almacenar en el mapa de variables.

    Para casos más elaborados podremos instanciar un parser, aplicarle
    opciones y luego procesar la entrada (ejemplo 2).
  \end{block}
  \pause
  \begin{block}{Acceso}
    Las opciones, una vez parseadas, se guardan en el mapa de
    variables, al que se puede acceder como un mapa cualquiera
    (casteando los valores).
  \end{block}
\end{frame}

\subsection{boost::foreach}

\begin{frame}[fragile]{boost::foreach}
  \begin{block}{}
    \texttt{boost/foreach.hpp}
  \end{block}
  \begin{block}{boost::foreach}
    Esta biblioteca hace proporciona exactamente lo que su nombre
    indica: una forma fácil de iterar sobre un contenedor usando una
    estructura \texttt{foreach}, presente en infinidad de lenguajes.

    Proporciona una macro \texttt{BOOST\_FOREACH} que recibe dos
    parámetros: el objeto en el que guardar el valor, y el contenedor
    a recorrer:

\begin{verbatim}
std::string cadena("Hola mundo");
BOOST_FOREACH(char c, cadena) {
    std::cout << c;
}
\end{verbatim}
  \end{block}
\pause
  \begin{block}{Tipos \textit{foreacheables}}
    \begin{itemize}
    \item Contenedores STL.
    \item Arrays.
    \item Cadenas terminadas en \texttt{NULL}.
    \item Extendible a tipos del usuario.
    \end{itemize}
  \end{block}
\end{frame}

\begin{frame}[fragile]{}
  \begin{block}{Recorrido inverso}
    También existe \texttt{BOOST\_REVERSE\_FOREACH}, para iterar del
    final al principio.
  \end{block}
  \pause
  \begin{block}{Poniéndolo bonito}
    Para evitar tener que usar la macro con mayúsculas, los \texttt{define} recomendados por \texttt{boost} son:
\begin{verbatim}
#define foreach         BOOST_FOREACH
#define reverse_foreach BOOST_REVERSE_FOREACH
\end{verbatim}
  \end{block}
  \pause
  \begin{block}{Problemas conocidos}
    \begin{itemize}
    \item Los tipos de datos \textbf{con comas} suelen dar problemas,
      ya que teóricamente la única coma que debe haber es la que
      separa los dos argumentos del \texttt{foreach}. Para usar tipos
      con comas (como \texttt{std::pair}), lo mejor es usar un typedef
      antes.
    \item \texttt{boost::foreach} está implementado usando
      internamente \textbf{iteradores}, por lo que la adición o
      eliminación de elementos al contenedor que estemos recorriendo
      con \texttt{foreach} da lugar a la invalidación de los
      iteradores previamente definidos, lo que suele conllevar un
      comportamiento errático en tiempo de ejecución.
    \end{itemize}
  \end{block}
\end{frame}

\subsection{boost::lexical\_cast}

\begin{frame}[fragile]{boost::lexical\_cast}
  \begin{block}{}
    \texttt{boost/lexical\_cast.hpp}
  \end{block}
  \begin{block}{Background}
    ¿Cuántas veces hemos buscado en Google \textit{``string to int''}?
    ¿Cuántas veces has mirado la sintaxis de un \texttt{stringstream}?
    ¿Sabes qué significa la 'a' en \texttt{atoi}?

    \medskip

    Las conversiones léxicas son un tema peliagudo y recurrente,
    \texttt{boost::lexical\_cast} ofrece una solución genérica que
    funciona, fácil de usar y recordar, y con un impacto mínimo en
    rendimiento y tamaño del ejecutable.
  \end{block}
\pause
  \begin{block}{Sintaxis}
    Básicamente, la sintaxis es la siguiente:

\begin{verbatim}
tipoDestino lexical_cast<tipoDestino>(tipoOrigen & valor);
\end{verbatim}

    Esto, señores, es \textit{magia potagia}. Bueno, realmente no:
    Básicamente, de forma interna se vierte el valor a convertir en un
    flujo, que se utiliza para rellenar el tipo destino.
  \end{block}
\end{frame}

\begin{frame}{Ampliación y errores}
  \begin{block}{}
    Puede utilizarse con cualquier tipo de datos, siempre y cuando se cumpla lo siguiente:
    \begin{itemize}
    \item El \texttt{tipoOrigen} debe tener definido el \texttt{operador <<}
      para flujos.
    \item El \texttt{tipoDestino} debe tener definido el \texttt{operador >>}
      para flujos.
    \item El \texttt{tipoDestino} debe poderse construir por copia y tener
      constructor por defecto.
    \end{itemize}
  \end{block}

  \pause

  \begin{block}{Excepciones}
    Cuando no se puede hacer la conversión, se lanza una excepción del tipo \texttt{bad\_lexical\_cast : public std::bad\_cast}.
  \end{block}

  \pause

  \begin{block}{Casts entre tipos numéricos}
    Para convertir entre tipos numéricos, en especial tipos de
    diferente ancho, es mejor usar \texttt{boost::numeric\_cast}.
  \end{block}
\end{frame}

\subsection{boost::smart\_pointers}

\begin{frame}{Introducción}
  \begin{block}{Smart pointers}
    Un \texttt{smart pointer} es una clase de puntero que ofrece una
    serie de funcionalidades extra en comparación con los punteros
    tradicionales. La funcionalidad más efectiva es la \textbf{gestión
    automática de la memoria} ocupada por el puntero.
  \end{block}
  \pause
  \begin{block}{Smart pointers en Boost}
    Boost nos ofrece varios tipos de smart pointers, los más importantes son:
    \begin{description}
    \item[Scoped pointer] Puntero no copiable.
    \item[Shared pointer] Puntero copiable.
    \item[Weak pointer] Observador de shared pointer.
    \end{description}
    Boost también ofrece smart pointers a arrays, pero suelen dar
    problemas y se recomienda usar un vector de shared pointers.
  \end{block}
\end{frame}

\begin{frame}[fragile]{Scoped pointer}
  \begin{block}{}
    \texttt{boost/scoped\_ptr.hpp}
  \end{block}
  
  \begin{block}{Scoped pointer}
    Un scoped pointer es una clase de smart pointer que no puede
    copiarse, por lo que solo puede existir un punto de acceso al
    objeto al que apunta.

    \medskip

    Cuando el puntero sale del ámbito, el objeto se destruye y la memoria se libera.
  \end{block}
  \pause
  \begin{block}{Sintaxis}
\begin{verbatim}
boost::scoped_ptr<MiClase> MiPuntero (new MiClase(1));
MiPuntero.reset(new MiClase(2));
\end{verbatim}
    Se puede acceder al contenido usando el operador *, acceder a la
    dirección con \& y acceder al puntero en bruto con el método \texttt{get()}.
  \end{block}
\end{frame}

\begin{frame}[fragile]{Shared pointer}
  \begin{block}{}
    \texttt{boost/shared\_ptr.hpp}
  \end{block}
  
  \begin{block}{Shared pointer}
    Un shared pointer es un tipo de smart pointer que guarda un
    contador de referencias al objeto al que apunta. Cada vez que se
    hace una copia del puntero, se aumenta el contador. Cuando se
    destruye uno de los shared pointer, el contador disminuye. 

    \medskip

    Cuando el contador llega a cero, quiere decir que no hay más
    punteros apuntando al objeto, por lo que éste puede destruirse y
    liberar la memoria que ocupa. Todo esto se hace de forma
    transparente al usuario.

  \end{block}
  \pause
  \begin{block}{Sintaxis}
\begin{verbatim}
boost::shared_ptr<MiClase> MiPuntero (new MiClase(1));
boost::shared_ptr<MiClase> OtroPuntero = MiPuntero;
MiPuntero.reset(new MiClase(2));
\end{verbatim}
  \end{block}
\end{frame}

\begin{frame}[fragile]{Funciones auxiliares para shared pointers}
  \begin{block}{make\_shared - \texttt{boost/make\_shared.hpp}}
    \texttt{make\_shared} es una alternativa a usar el operador
    \texttt{new} a la hora de inicializar un shared pointer. Se trata
    de una función paramétrica que nos devuelve un shared pointer a un
    objeto recién creado.
{\small
\begin{verbatim}
boost::shared_ptr<string> X (new string("argumentos");
// en lugar de new, usamos:
boost::shared_ptr<string> X = boost::make_shared<string>("argumentos");
\end{verbatim}
} El uso de \texttt{make\_shared} es más eficiente que \texttt{new},
ya que de una vez se hace la asignación de memoria para el propio
puntero y para su contenido, en lugar de crear el contenido y luego
hacer un puntero que apunte a él, como ocurre con \texttt{new}.
  \end{block}
\end{frame}

\begin{frame}[fragile]{Funciones auxiliares para shared pointers}
\begin{block}{\texttt{enable\_shared\_from\_this.hpp}}
  Define una clase paramétrica \texttt{enable\_shared\_from\_this},
  que se usa como clase base para permitir obtener un shared pointer
  al objeto actual desde una función miembro, utilizando la función
  \texttt{shared\_from\_this()}.

  \medskip

  Es imprescindible que en algún lado del código ya exista un shared
  pointer apuntando al objeto.    
  \end{block}

  \begin{block}{}
    Ver ejemplo \texttt{shared\_aux.cpp}
  \end{block}
\end{frame}

\subsection{boost::bind y boost::function}
\begin{frame}[fragile]{Definiciones}
  \begin{block}{Objeto-función}
    Un \textbf{objeto-función} es una instancia de una clase que tiene
    el operador () sobrecargado, de modo que es posible utilizar el
    objeto como si de una función se tratara.

    \medskip

    La mayoría de los algoritmos de la STL son capaces de recibir
    objetos-función para personalizar su funcionamiento, por ejemplo:
{\small
\begin{verbatim}
struct objeto{
    void operator()(int & a){  cout << "Núm: " << a << endl; }
};

int main(){
    vector<int> v;  
    v.push_back(1);  
    v.push_back(2);
    std::for_each(v.begin(), v.end(), objeto());
}
\end{verbatim}
}
  \end{block}
\end{frame}

\begin{frame}[fragile]{Definiciones}
  \begin{block}{}
    \texttt{boost/bind.hpp}\\
    \texttt{boost/function.hpp}
    
  \end{block}

  \begin{block}{boost::function}
    \texttt{boost::function} es un contenedor para objetos función;
{\small
\begin{verbatim}
struct objeto{
    void operator()(int a) { cout << "Núm: " << a << endl; }
};

boost::function<void(int)>f = objeto();  
\end{verbatim}
}
  \end{block}

  \begin{block}{boost::bind}
    \texttt{boost::bind} es una generalización de
    \texttt{std::bind1st} y \texttt{std::bind2nd}. Nos permite
    \textit{bindear} una serie de argumentos a cualquier cosa que
    \textit{tenga pinta de función}: punteros a función, objetos
    función, y punteros a función-miembro y devolver un objeto función
    que encapsula la llamada.

    \bigskip

    Es difícil de entender, así que lo mejor es verlo con ejemplos.
  \end{block}
\end{frame}

\begin{frame}[fragile]{Ejemplos de boost:bind}
  Imaginad que tenemos una función con 6 argumentos, y el 90\% de las
  veces que la ejecutamos, cinco de esos argumentos son los
  mismos. ¿No nos interesaría tener algún mecanismo para generalizar
  esas llamadas? Con \texttt{bind} y \texttt{function} \textbf{¡podemos!}.

  Simplemente tendríamos que llamar a \texttt{bind}, decirle qué
  función queremos bindear, y cómo tratar los argumentos.
  \begin{itemize}
  \item Podemos darle valores a los argumentos escribiendo
    directamente el valor.
  \item Podemos reorganizar los argumentos, por ejemplo haciendo que
    el tercer argumento de la función sea el primero del binding. Para
    ello se usan los \textit{placeholders} \_1, \_2, \_3...
  \end{itemize}
  La forma de almacenar el resultado es con un objeto
  \texttt{boost::function}.
\end{frame}

\begin{frame}[fragile]{Ejemplos de boost:bind}
  \begin{block}{}
\begin{verbatim}
float fun(int a, int b, int c, int d, int e, int f){
    return a + b + c + d + e + f;
}

function<float(int)> f = bind(fun, 10, 8, _1, 12, 43, 93);
cout << f(2) << endl; // equivale a
cout << fun(10, 8, 2, 12, 43, 93) << endl;
\end{verbatim}
  \end{block}

  \begin{block}{Otros casos}
    \begin{itemize}
    \item Bindear a función objeto.
    \item Bindear a función miembro de una clase.
    \item Composición de funciones.
    \end{itemize}
    
  \end{block}

\end{frame}

\subsection{boost::any}

\begin{frame}[fragile]{Introducción}
  \begin{block}{}
    \texttt{boost/any.hpp}
  \end{block}

  \begin{block}{Definición}
    \texttt{boost::any} es una clase que nos permite almacenar y
    recuperar de forma segura elementos de cualquier tipo.  
    
    \medskip
    
    Se podría equiparar a usar un \texttt{void *}, pero con los
    beneficios de ser \textit{type-safe}, ya que a la hora de
    recuperar el valor, es necesario saber el tipo de datos que
    estamos recuperando.
  \end{block}

\end{frame}

\begin{frame}[fragile]{Sintaxis}
  \begin{block}{Sintaxis}
    La sintaxis de \texttt{any} es muy sencilla. Podemos asignar valores directamente:
    {\small
\begin{verbatim}
boost::any cualquiera = 5;
cualquiera = "Cadena";
\end{verbatim}
    }

    Para acceder a los valores debemos conocer su tipo:
{\small
\begin{verbatim}
boost::any valor = 5;
cout << boost::any_cast<int>(valor) << endl;
\end{verbatim}
}

¡Ojo! \texttt{any\_cast} lanzará una excepción si el tipo es
incorrecto. En cambio, si trabajamos con punteros, no lanzará
excepciones, sino que devolverá un puntero nulo.  {\small
\begin{verbatim}
boost::any num = 5;
string * a;
if (!(a = boost::any_cast<string>(&num))){
    cout << "No es una cadena" << endl;
}
\end{verbatim}
}
  \end{block}
\end{frame}

\subsection{boost::String Algorithms}
\begin{frame}[fragile]{String Algorithms}
  \begin{block}{Algoritmos para cadenas}
    Boost incluye una vasta colección de algoritmos para la
    transformación y el manejo de cadenas. Por regla general se trata
    de funciones y predicados muy sencillos, que realizan búsquedas, comprobaciones y transformaciones de cadenas:
    \begin{itemize}
    \item Cambios de mayúsculas y minúsculas.
    \item Borrado de espacios laterales (\texttt{trailing whitespace}).
    \item Comparaciones y comprobaciones de inicios y finales.
    \item Algoritmos de búsqueda.
    \item Algoritmos de borrado y reemplazo.
    \item Splitting y joining (muy útil).
    \end{itemize}
\url{http://www.boost.org/doc/libs/1_43_0/doc/html/string_algo/quickref.html}
  \end{block}

  \begin{block}{}
    {\small
\begin{verbatim}
#include <boost/algorithm/string/join.hpp>

vector<string> cadenas;
cadenas.push_back("uno");
cadenas.push_back("dos");
cout << boost::algorithm::join(cadenas, ",") << endl;
\end{verbatim}
    }
  \end{block}
\end{frame}



\subsection{Bibliotecas omitidas}

\begin{frame}{Otras bibliotecas}
  \begin{block}{Bibliotecas que se han quedado fuera}
    Como el tiempo del taller es limitado, las siguientes bibliotecas
    no las podremos ver pero son interesantes/importantes.
    \begin{itemize}
    \item \texttt{boost.serialization} - Serialización
    \item \texttt{boost.spirit} - Generación de parsers para
      gramáticas regulares en forma EBNF.
    \item \texttt{boost.signals} - Framework para implementar el
      patrón observer con un modelo señales y slots similar a Qt.
    \item \texttt{boost.filesystem} - Lectura y escritura del sistema de
      ficheros y gestión de rutas. Multiplataforma.
    \item \texttt{boost.asio} Librería para comunicación asíncrona.
    \item \texttt{boost.multiindex} - Contenedor que permite el uso de
      varios campos como índices y más.
    \end{itemize}
  \end{block}
\end{frame}
\end{document}

